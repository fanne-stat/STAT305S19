\documentclass[11pt]{article}
\textheight 9in
\textwidth 6.5in
\oddsidemargin 0in
\topmargin 0in
\headheight 1in
\headsep 0in
\usepackage{color}
\usepackage{lastpage}
\usepackage{fancyhdr}
\pagestyle{fancy}
\cfoot{}
\rhead{\thepage\ of \pageref{LastPage}}
% \renewcommand{\headrulewidth}{0pt}
\usepackage[top = 4em, headsep=.2in]{geometry}
\begin{document}

\noindent{{\textbf{\large Spring 2020  \hfill Stat 305 (Section 4)
\hfill %Page \thepage   { of }  \pageref{lastpage}
Quiz 1}}

\vspace{2em}

\noindent\framebox(200, 50){\vspace{3em}\hspace{2em}\bf Name: \hspace{6cm}}

\bigskip

\noindent\emph{Total points for the exam is 50. Points for
individual questions are given at the beginning of each problem.
Show all your calculations clearly. Put final answers in the box at
the right (except for the diagrams!).}


\bigskip


\noindent {\bf 1.}\hfill[8+6+6+6=26 points] \\

\noindent The following are the closing prices of Facebook from Jan 21, 2020 to Jan 31, 2020:\\

\hspace{1.5cm}$221.44, 221.32, 218.76, 217.94, 214.87, 217.79, 223.23, 209.53, 201.91$.

\begin{enumerate}
\item[(a)] Find the 0.82 quantile, the median, first and third quartile for
the above data.

\hfill \fbox{ \textcolor[rgb]{1.00,1.00,1.00}{$\bigcap$} \hskip
-0.4cm $Q(0.82)=$ \hspace{2cm}}

\hfill \fbox{$ Med.=$ \hspace{2cm}}

\hfill \fbox{1st Quart.=  \hspace{2cm}}

\hfill \fbox{3rd Quart.=  \hspace{2cm}}


\vskip 6cm

\item[(b)] Give the coordinates (on a regular
graph paper) of the upper right and lower left point that would
appear on a normal plot of the data.

\hfill \fbox{upper right point = $(\phantom{xxxxxxx}, \phantom{xxxxxxx})$ }

\hfill \fbox{lower left point = $( \phantom{xxxxxxx}, \phantom{xxxxxxx})$}


\vskip 2cm
\newpage
\item[(c)] Draw a boxplot for this data. Carefully label numbers on
the plot

\vskip 12cm

\item[(d)] Find the sample mean and standard deviation for this
data. Show calculations.

\hfill \fbox{ \textcolor[rgb]{1.00,1.00,1.00}{$\bigcap$} \hskip
-0.3cm $\bar x=$ \hspace{2cm}}

\hfill \fbox{ \textcolor[rgb]{1.00,1.00,1.00}{$\bigcap$} \hskip
-0.3cm $ s=$ \hspace{2cm}}


\end{enumerate}


\newpage
\noindent {\bf 2.}\hfill[8$\times$3=24 points] \\For each of the
following questions, choose only one (the best) answer. No credit
will be given if more than one is chosen.

\begin{enumerate}
\item[(a)] Which of the following is the best numerical summary that is insensitive to outliers if we want to assess the precision of a measurement system? 

(A) sample variance \hspace{0.5cm} (B) IQR \hspace{0.5cm}(C) sample mean\hspace{0.5cm} (D) median

\vskip -0.7cm \hspace{6in}\fbox{
\textcolor[rgb]{1.00,1.00,1.00}{$\bigcap$}}

\item[(b)] Which of the following best describes the methods for handling
extraneous variables:

(A) blocking and replication \hspace{0.5cm}(B) randomization and
replication \\ (C) randomization and blocking \hspace{0.5cm}(D)
randomization, blocking, and replication

\vskip -0.7cm \hspace{6in}\fbox{
\textcolor[rgb]{1.00,1.00,1.00}{$\bigcap$}}


\item[(c)] For a complete factorial study with 8 factors, each with 3
levels, the number of observations is at least

(A) 1024 \hspace{0.5cm}(B) 6561 \hspace{0.5cm}(C) 512
\hspace{0.5cm}(D) none of these

\vskip -0.7cm \hspace{6in}\fbox{
\textcolor[rgb]{1.00,1.00,1.00}{$\bigcap$}}


\item[(d)] For a $3 \times 3$ full factorial study with two factors A and B, where
A has three levels (low, medium and high) and B has three levels (low, medium,
and high), the nine experimental runs are labeled as:

No. 1: A low B low, \hspace{0.5cm} No. 2: A low B medium,
\hspace{0.5cm} No. 3: A low B high, \\ 
No. 4: A medium B low,
No. 5: A medium B medium, No. 6: A medium B high, \\ No. 7: A high B low,
\hspace{0.5cm} No. 8: A high B medium, \hspace{0.1 cm}and
\hspace{0.1cm} No. 9: A high B high.

\noindent Based on the following random digits

\hspace{3.5cm}$97437 \; 52922 \; 80739 \; 59178 \; 50628$

Which experiment should be done last?

(A) No. 6 \hspace{0.5cm}(B) No. 7 \hspace{0.5cm}(C) No. 8
\hspace{0.5cm}(D) No. 9

\vskip -0.7cm \hspace{6in}\fbox{
\textcolor[rgb]{1.00,1.00,1.00}{$\bigcap$}}


\item[(e)] Based on the following random digits

\hspace{3.5cm}$ 61017 \;51652 \; 40915 \; 94696 \; 67843 \; 58009$

the second widget selected from 99 widgets labeled 1,2,\ldots ,99 is

(A) 51 \hspace{0.5cm}(B) 10 \hspace{0.5cm}(C) 1\hspace{0.5cm}(D)
75

\vskip -0.7cm \hspace{6in}\fbox{
\textcolor[rgb]{1.00,1.00,1.00}{$\bigcap$}}


\item[(f)] In a series of experiments to study the purity of a chemical
product, the effect of reactant A on the purity was of primary
interest and three levels of A were used in the experiments. We also know these experiments are done in two different labs. The variable lab is a

(A) concomitant variable \hspace{0.5cm} (B) controlled variable
\\ (C) blocking variable \hspace{0.5cm}(D) experimental variable

\vskip -0.7cm \hspace{6in}\fbox{
\textcolor[rgb]{1.00,1.00,1.00}{$\bigcap$}}



\item[(g)] What is the relationship between the median and mean of the
	data set that is exponentially distributed and skewed to right?

(A) median $<$ mean \hspace{0.5cm}(B) median $>$ mean \hspace{0.5cm}

(C) median = mean
\hspace{0.5cm}(D) all above are possible

\vskip -0.7cm \hspace{6in}\fbox{
\textcolor[rgb]{1.00,1.00,1.00}{$\bigcap$}}



\item[(h)] If one wished to assess if a data set is normally distributed, one
should use

(A) a dot diagram \hspace{0.5cm}(B) a histogram \hspace{0.5cm}(C) a
boxplot \hspace{0.5cm}(D a normal Q-Q plot

\vskip -0.7cm \hspace{6in}\fbox{
\textcolor[rgb]{1.00,1.00,1.00}{$\bigcap$}}


%
\label{lastpage}

\end{enumerate}
\end{document}

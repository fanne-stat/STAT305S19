\documentclass{article}
\usepackage[hmargin = 1in]{geometry}
\usepackage{xcolor}
\begin{document}





\begin{center} \LARGE
Homework 2 
\end{center}
\begin{center} \Large
Due January 30, 2020 at 11:59 PM 
\end{center}



\begin{enumerate}
	\item P. 47: 1 (5 points)

		{\color{red} 
		Possible controlled variables: operator, lauch angle, launch
	force, paper clip size, paper manufactuer, plane constructor,
distance measureer, and wind. The response is Flight Distance and the
experimental variables are Design, Paper Type, and Loading Condition.
Concomitant variables might be wind speed and direction (if these cannot be
contrlled), ambient temperature, humidity and atmospheric pressure.}
	\item P. 47: 2 (5 points)

		{\color{red}
			Advantage: may reduce baseline variation (background
			noise) in the response, making it easier to see the
			effects of factors. Disadvantage: the variable may
			fluctuate in the real world, so controlling it makes
			the experiment more artificial --- it will be harder
			to generalize conclusions from the experiment to the
			real world.
		}

	\item P. 64: 1 (5 points)
		{\color{red}
			Label the widgets 1, 2, ..., 491 (can also label 0, 1,
			2, ..., 490). Choose the widgets
			labeled 121, 405, 91, 134, 464, 313, 249, 141. 
		}

	\item P. 64: 2 
			\begin{enumerate}
				\item (5 points)
					{\color{red} 
						Possible responses: volume of poped
					corn, numer of unpoped kernels, and
				tasteof poped corn.}
			\item (5 points)
					{\color{red} Time poped (short vs. long) and Poping
					Method (frying vs. hot air poping) are
					two possible factors. A $2 \times 2$
					factorial data structure would result
					from choosing two levels for each
					factor (as was done above). and testing
					all 4 factor level combinations:
					\begin{center}
						\begin{tabular}{ll}
						Time & Poping Method \\ \hline
						short & frying \\
						long & frying \\
						short & hot air \\
						long & hot air
						\end{tabular}
				\end{center}}
			\item (5 points)
					{\color{red} You can randomly assign one-forth of
					the available kernels to each
					factor-level combination. You could
					randomize the order in which each test
					is performed. If the measurement of the
					response is subject to measurement
					error or time effects, you might also
					randomize the order in which each batch
				is measured.}
			\item (5 points)
					{\color{red} If there will be replications, there
					may not be enough popcorn in one
					package to supply the entire
					experiment; it may be necessary to use
					2 or more packages of corn. In this
					case, package could be treated as
					blocking factor. For each package, one
					test could be performed for each
				factor-level combination.}

			\end{enumerate}
		
	\item P. 65: 8
		
			\begin{enumerate}
				\item 
					(5 points) {\color{red} 
						See P. 24. 8(b) for the factors and
					levels. Two possible responses would
					be flatness and concentricity (other
				reasonable responses are also OK).}

					(5 points)
					{\color{red}
					Replication dictates that at least one
					of the 8 factor-level combinations
					given in 8(b) be run at least twice.
					One possibility is to run each
					factor-level combination twice, for
				total of 16 runs.

					\begin{tabular}{ccccccc}
						Test Label & Test Order &
						Vendor & Heating Time &
						Cooling Method & Flatness &
						Concentricity \\ \hline
						1 & & 1 & short & 1 & & \\
						\hline
						2 & & 1 & short & 1 & & \\
						\hline
						3 & & 2 & short & 1 & & \\
						\hline
						4 & & 2 & short & 1 & & \\
						\hline
						5 & & 1 & long & 1 & & \\
						\hline
						6 & & 1 & long & 1 & & \\
						\hline
						7 & & 2 & long & 1 & & \\
						\hline
						8 & & 2 & long & 1 & & \\
						\hline
						9 & & 1 & short & 2 & & \\
						\hline
						10 & & 1 & short & 2 & & \\
						\hline
						11 & & 2 & short & 2 & & \\
						\hline
						12 & & 2 & short & 2 & & \\
						\hline
						13 & & 1 & long & 2 & & \\
						\hline
						14 & & 1 & long & 2 & & \\
						\hline
						15 & & 2 & long & 2 & & \\
						\hline
						16 & & 2 & long & 2 & & \\
						\hline
				\end{tabular}}
				\item (5 points)
					{\color{red}
					For the scenario in (a), you should use
					16 slips of paper. Each slip
					corresponds to a run. Order the runs in
					the same order as their corresponding
					slips are picked from the hat. Avoid
					placing the slips into the har in any
					special order, and mix the slips well
					before picking them. All slips should
					be physicalling identical so that teh
				selection order is completely random.}
			\end{enumerate}
		
\end{enumerate}




%\newpage 
%\nocite{*}
%\bibliographystyle{plainnat} 
%\bibliography{}
\end{document}

\documentclass{article}
\usepackage[hmargin = 1in]{geometry}
\usepackage{xcolor}
\begin{document}





\begin{center} \LARGE
Homework 1 
\end{center}
\begin{center} \Large
Due January 23, 2020 at 11:59 PM 
\end{center}



\begin{enumerate}
	\item P. 13: 1 (5 points)

		{\color{red} Observational study --- you might be intereted in assessing the
		job satisfaction of a large number of manufacturing workers;
		you cound administer a survey to measure various dimensions of
		job satisfaction. Experimental studu --- you might want to
		compare several different job routing schemes to see which one
	achieves the greatest throughput in a job shop.}

	\item P. 13: 2 (5 points)

		{\color{red}
		Qualitative data --- rating the quality of batches of ice
		cream as either poor, fair, good, or exceptional.
		Quantitative data --- measuring the time (in hours) it takes
		for each of 1000 integrated circuit chips to fail in a
		high-stress environment.
		
		}
	\item P. 14: 6 (5 points)
		{\color{red}
		Variables can be manipulated in an experiment. If changes in
		the response coincide with changes in factor levels, it is
		usually safe to infer the change in the factor caused the
		changes in the response (as long as other factors have been
		controlled and there is no source of bias). There is no
		control or manipulation in an observational study. Changes in
		the response may coincide with changes in another variable, but
		there is always the possibility that a third variable is
		causing the correlation. It is therefore risky to infer a
		cause-and-effect relationship between any variable and the
		response in an observational study. 

		(Or one can just say there is no treatment variable in an
		observatinal study.)
		}
	\item P. 19: 1 (5 points)
		{\color{red} Even if a measurement system is accurate and precise, if it is
		not truly measuring the desired dimension or characteristic,
		then the measurements are useless. If a measurement system is
		valid and accurate, but imprecise, it may be useless because
		it produces too much variability (and cannot be corrected by
		calibration). If a measurement system is valid and precise, but
		inaccurate, it might be easy to make it accurate (and then
	useful) by calibrating it to a standard.}
	\item P. 24: 8 (5+5 points)
		{\color{red}
		\begin{enumerate}
			\item Rockwell hardness: multivariate (bivariate),
				repeated measures (paired), quantitative
				data. Flatness: univariate, qualitative data.
			\item There are many possibilities. Possible factors
				are Vendor, Material, Heating Time, Heating
				Temperature, Cooling Methdo, and Furnace
				Atmosphere Condition. You could choose any
				number of levels for each factor. If you
				choose Vendor (1 vs 2), Heating Time (short vs
				long), and Cooling Method (1 vs 2), the
				factor-level combinations are given below.

				\begin{center}
					\begin{tabular}{ccc}
					Vendor & Heating Time & Cooling Method\\
					\hline
					1 & short & 1\\
					2 & short & 1\\
					1 & long & 1\\
					2 & long & 1\\
					1 & short & 2\\
					2& short & 2\\
					1& long & 2\\
					2 & long & 2
					\end{tabular}
				\end{center}
		\end{enumerate}
		}
\item
\begin{enumerate}
	\item
		{\color{red} \begin{itemize}
\item Population: All Type I Diabetes-afflicted rats. 

\item Sample: the 18 diabetes-afflicted in the study.
\end{itemize}}
\item
{\color{red} Experiment: the investigators applied the medication (treatment) themselves while keeping experimental conditions constant for each rat (and thus for each level of treatment).}
\item
{\color{red}
\begin{itemize}
\item Treatment: medication level
\item Response: improvement in rat fitness 
\end{itemize}}

\item
{\color{red} There are 2 experimental groups: one with the rats who were given the medication, one with the rats who were not given the medication. }
\end{enumerate}
\end{enumerate}




%\newpage 
%\nocite{*}
%\bibliographystyle{plainnat} 
%\bibliography{}
\end{document}

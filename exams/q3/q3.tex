\documentclass[11pt]{article}
\textheight 9in
\textwidth 6.5in
\oddsidemargin 0in
\topmargin 0in
\headheight 1in
\headsep 0in
\usepackage{color}
\usepackage{lastpage}
\usepackage{graphicx}
\usepackage{fancyhdr}
\pagestyle{fancy}
\cfoot{}
\rhead{\thepage\ of \pageref{LastPage}}
% \renewcommand{\headrulewidth}{0pt}
\usepackage[top = 4em, headsep=.2in]{geometry}
\usepackage{textcomp}
\usepackage{enumitem}

\begin{document}

\noindent{{\textbf{\large Spring 2020  \hfill Stat 305 (Section 4)
\hfill %Page \thepage   { of }  \pageref{lastpage}
Quiz 3}}

\vspace{2em}

\noindent\framebox(200, 50){\vspace{3em}\hspace{2em}\bf Name: \hspace{6cm}}

\bigskip

\noindent\emph{Total points for the exam is 50. Points for
individual questions are given at the beginning of each problem.
Show all your calculations clearly. Put final answers in the box at
the right (except for the diagrams!).}

\noindent{\bf Some quantiles might be used:} $z_{0.975} = 1.96,\, z_{0.95} = 1.64$.


\bigskip


\noindent {\bf 1.}\hfill[5+5+5=15 points] \\
%
\noindent Suppose that a factory manufactures a rod-shaped part. The lengths of this kind of part are normally distributed with mean 7.50 cm and standard deviation 0.015 cm (\emph{round to 3 decimal places}). 


\begin{enumerate}
\item[(a)] Evaluate the probability that the length of the next manufactured part is between 7.49 cm and 7.51 cm.

\hfill \fbox{ \textcolor[rgb]{1.00,1.00,1.00}{$\bigcap$} \hskip
-0.4cm probability= \hspace{2cm}}
%
%\hfill \fbox{ \textcolor[rgb]{1.00,1.00,1.00}{$\bigcap$} \hskip
%-0.4cm Equation 2: $\; \hat y =$ \hspace{10cm}}


\vskip 6cm




\item[(b)] Evaluate the probability that a the mean length of the next 30 parts is between 7.49 cm and 7.51 cm (\emph{round to 4 decimal places}).

\hfill \fbox{ \textcolor[rgb]{1.00,1.00,1.00}{$\bigcap$} \hskip
-0.4cm probability= \hspace{2cm}}

\vskip 2.5cm
\clearpage
\item[(c)] If the factory want to control that 95\% of the parts' lengths are between 7.49 cm and 7.51 cm, what manufaturing precision (as measured by standard deviation $\sigma$) is required if the mean length is still 7.5 cm (\emph{round to 4 decimal places})?

\hfill \fbox{ \textcolor[rgb]{1.00,1.00,1.00}{$\bigcap$} \hskip
-0.4cm  s.d.= \hspace{2cm}}

\vskip 7cm


\end{enumerate}


\noindent {\bf 2.}\hfill[5 points] \\
%
Suppose that $X$, $Y$, and $W$ are independent random variables each
with  mean   5 and standard deviation 1. Find the mean and
standard deviation of $(X + 3Y- \sqrt{2} W)$ (\emph{round to 4 decimal places}).\\

\hfill \fbox{ \textcolor[rgb]{1.00,1.00,1.00}{$\bigcap$} \hskip
-0.4cm mean= \hspace{2.7cm}}\\

\hfill \fbox{ \textcolor[rgb]{1.00,1.00,1.00}{$\bigcap$} \hskip
-0.4cm s.d.= \hspace{3cm}}

%\begin{tabular}{|r|c|c|c|c|c|}
%  \hline
%  % after \\: \hline or \cline{col1-col2} \cline{col3-col4} ...
%  x & $-4$ & $-2$ & $-1$ & $3$ & $4$ \\
%  \hline
%  f(x) & 0.2 & 0.3 & 0.1 & 0.1 & 0.3 \\
%  \hline
%\end{tabular}

\newpage

\noindent {\bf 3.}\hfill[4 + 6 points] \\
%
Suppose that the weights for each bottle of the bottled water from a factory are normally distributed with mean 500 g and standard deviation 5 g. 40 bottles of water will be packed in a box.


\begin{itemize}
	\item[(a)] Find the mean and variance for the weight of a box of bottled water (ignore the weight of the box). 

	\hfill \fbox{ \textcolor[rgb]{1.00,1.00,1.00}{$\bigcap$} \hskip
-0.4cm mean= \hspace{3.5cm}}\\

\hfill \fbox{ \textcolor[rgb]{1.00,1.00,1.00}{$\bigcap$} \hskip
-0.4cm variance= \hspace{3cm}}
\vskip 7cm
\item[(b)] Find the probability that the weight of a box of bottled water is less than 19950 g (ignore the weight of the box) (\emph{round to 4 decimal places}). 

\hfill \fbox{ \textcolor[rgb]{1.00,1.00,1.00}{$\bigcap$} \hskip
-0.4cm probability= \hspace{2cm}}
\end{itemize}



%\hfill \fbox{ \textcolor[rgb]{1.00,1.00,1.00}{$\bigcap$} \hskip
%-0.4cm For equation 2:\hspace{3cm}}

\vskip 3.5cm

\clearpage
\noindent {\bf 4.}\hfill[5+5+10=20 points] \\
%
Some students measured the heights of 405 steel punches of a particular type. These were all from a single manufacturer and were supposed to have heights of 0.500 in. (The stamping machine in which these are used is designed to use 0.500 in. punches.) The students’ measurements had $ \bar{x} = 0.5004$ in. and $s = 0.0026$ in.
%
\begin{enumerate}

\item[(a)] Give a 95\% two-sided confidence interval for the mean height of the steel punch of this particular type (\emph{round to 5 decimal places}).




 \hfill \fbox{ \textcolor[rgb]{1.00,1.00,1.00}{$\bigcap$}
\hskip -0.4cm $C.I= (\hspace{2cm},\hspace{2cm})$ }

\vskip 8cm
\item[(b)] Give a 95\% upper confidence bound
for the mean height of the steel punch of this particular type (\emph{round to 5 decimal places}).



 \hfill \fbox{ \textcolor[rgb]{1.00,1.00,1.00}{$\bigcap$}
\hskip -0.4cm upper conf. bd = \hspace{2cm} }

\vskip 1cm
\clearpage
\item[(c)] Test if the mean height of the steel punch of this particular type differs from the desired height of 0.500 in. with \textbf{p-value}. {\bf Show all
the steps of the testing procedure.}


\end{enumerate}

%
\label{lastpage}


\end{document}
